\documentclass[]{scrreprt}
\usepackage{amsmath,amsfonts,graphicx}
\usepackage{multirow}
\usepackage{pslatex}
\usepackage{tabularx}
\usepackage{comment}
\usepackage{xspace}
\usepackage{array}

\usepackage{hyperref}

\usepackage{caption}
\DeclareCaptionFont{white}{\color{white}}
\DeclareCaptionFormat{listing}{\colorbox{gray}{\parbox{\textwidth}{#1#2#3}}}

\graphicspath{
{figures/}
}

\newcommand{\uo}{\mbox{UO\textsubscript{2}}\xspace}

\setcounter{secnumdepth}{3}


\begin{document}


\title{Richards Examples}
\author{Andy Wilkins \\
CSIRO}
\maketitle

\tableofcontents

%%%
\chapter{Introduction}
%%%

The Richards' equation describes slow fluid flow through a porous
medium.  This document outlines input-file examples for the Richards
MOOSE code, drawing mainly upon the test suite.  There are two other
accompanying documents: (1) The theoretical and numerical foundations
of the code, which also describes the notation used throughout this
document; (2) The test suite, which describes the benchmark tests used
to validate the code. 


\chapter{The examples}

Each example is located in the {\em test} directory, which has path
\begin{verbatim}
<install_dir>/trunk/elk/tests/richards
\end{verbatim}
or the {\em user} directory, which has path
\begin{verbatim}
<install_dir>/trunk/elk/doc/richards/user
\end{verbatim}

\section{Convergence criteria}

\section{Two-phase, almost saturated}

If a two-phase model has regions that are fully saturated with the
``1'' phase (typically this is water), then the residual for the ``2''
phase is zero.  This means the ``2''-phase pressure will not change in
those regions, potentially violating $P_{1}\leq P_{2}$.  If the ``2''
phase subsequently infiltrates to these regions, an initially crazy
$P_{2}$ might affect the results.  This sometimes also holds for
almost-saturated situations, depending on the exact simulation.

In these cases it is useful to add a penalty term to the residual to
ensure that $P_{1}\leq P_{2}$.  An example can be found in the tests
directory {\tt pressure\_pulse/pp22.i}.  The choice of the $a$
parameter is sometimes difficult: too big and the penalty term
dominates the Darcy flow; too small and the penalty term does
nothing.  In both cases, convergence is poor as the penalty term
switches on and off during the Newton-Raphson procedure.  The
documentation for {\tt RichardsPPenalty} describes how to set $a$ (run
MOOSE with a {\tt -\,-dump} flag).

The penalty term should {\em not} be used unless absolutely necessary
as it will lead to poorer convergence characteristics.  In many cases
it is not necessary.

\section{An excavation}

In the user directory {\tt excav/ex01.i} contains a single excavation,
and the associated mass flux and mass balance.







\end{document}

